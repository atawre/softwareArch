\documentclass[conference]{IEEEtran}
\IEEEoverridecommandlockouts
% The preceding line is only needed to identify funding in the first footnote. If that is unneeded, please comment it out.
\usepackage{cite}
\usepackage{amsmath,amssymb,amsfonts}
\usepackage{algorithmic}
\usepackage{graphicx}
\usepackage{textcomp}
\usepackage{xcolor}
\def\BibTeX{{\rm B\kern-.05em{\sc i\kern-.025em b}\kern-.08em
    T\kern-.1667em\lower.7ex\hbox{E}\kern-.125emX}}
\begin{document}

\title{The latest developments in advanced architectural patterns: a survey\\
{\footnotesize \textsuperscript{*}}
\thanks{Identify applicable funding agency here. If none, delete this.}
}

\author{\IEEEauthorblockN{Abhijit Taware}
\IEEEauthorblockA{\textit{Computer Science (Masters student)} \\
\textit{UNB}\\
Fredericton, Canada \\
ataware@unb.ca}
\and
\IEEEauthorblockN{Luigi Benedicenti}
\IEEEauthorblockA{\textit{ Computer Science (Dean)} \\
\textit{UNB}\\
Fredericton, Canada \\
luigi.benedicenti@unb.ca}
}

\maketitle

\begin{abstract}
With the advent of paradigm like microservices, containerized applications, cloud computing etc., the software developement process is being influenced heavily. This evolved the software developement and triggered the emergence of new architectural patterns. In this document, the common software architectural patterns are brifely discussed. The document further describes latest advancements in service oriented architecture, microservices, reactive programming and resilient software development.
\end{abstract}

\begin{IEEEkeywords}
SOA, microservice, reactive programming, resilient software
\end{IEEEkeywords}

\section{Architectural design patterns}
Large enterprise needs software that scales with ever chaging and increasing needs of the business. Selecting the right architecture before diving into the actual work is crucial to the success of the application and enterprise. This section explores various architectural patterns used in the industry. The pros and cons will be discussed for each of the pattern.

\subsection{Layered architecture}
It is the most common architecture style, that organize similar modules into horizontal layers. The layers are independent of others and inteact using exported APIs. An application can be designed using any number of layers. The network protocol stack is a good example of layered architecture. The in upper layer is transmitted to lower layers using encapsulated packets. A layer don't have to know the inner working of other layer and communication happens through a set of APIs exposed by each layer. Another example of bussiness application, that is divided into presentation, logic and data tiers. Following of some of the benefits offered by this architecture.
\begin{itemize}
\item Layers can be developed and tested independently.
\item Changes made in one layer doesn't affect the other layer, hence maintainable.
\item Low coupling and high cohesion
\item Lower layers have no dependency on higher layer and hence reusable.
\end{itemize}

The disadvantages can be summarized as follow.
\begin{itemize}
\item A change to any component may trigger a redeployment of the entire application.
\item Each layer can have separate physical deployment or an entire application can be replicated. It is too coarse grained from deployment perspective.
\item Communication across layers can be a performance bottleneck for certain applications.
\end{itemize}

\subsection{client-server architecture}
It consists of a server and multiple clients. The server keeps listening to the client requests. The server responds to any new client requests i.e., provides a service to those clients.
E.g. the encryption key control server \cite{hytrust} provides encryption keys to the requesting clients over network. This model is prone to denial of service attack. The scalability requires replicating the server components with load balancing, failover and failback mechanisms.
\subsection{Pipe and filter architecture} 
This approach is suitable for large applications that can be broken down to multiple steps. Each step refers to a filter. The filter applies a specific function to the data and can work asynchronously as well. The pipes refer to the connectors between these filters. The output of one filter serves as an input for the next filter on the pipeline. The common example is Unix pipes. 
\begin{itemize}
\item Adding a new step is easy by adding a new filter and adding it to existing pipe stream.
\item It is easier to reuse of filters doing generic actions.
\item Promotes concurrency of different filters do not depend on each other.
\item The errors gets propogated across the filters, which is a downside of this architecture.
\item A broken filter leads to a complete broken pipe.
\end{itemize}

\subsection{Peer to peer architecture}
A peer-to-peer (P2P) architecture consists of a decentralized network of peers i.e. nodes. Unline client-server architecture, a node in this architecture can act both as a server and a client.
The workload is split into small chunks that can be reassembled later, allowing peers to work simultaneously on a task. E.g. P2P file sharing, where a file is split into chunks that allows many chunks to be downloaded from diffrent peers at the same time.
\begin{itemize}
\item Need for centralized server is eliminated.
\item There is no single point of failure, unless the number of peers are too few.
\item The increase in number of peers can be handled easily i.e., scalable.
\item The model is prone to security issue, as an infected peer can affect the whole network.
\item Fairless garuntees are difficult to enforce as many leeches could benefit free riders.
\item Instant messaging, file sharing, collaboration apps use P2P architecture. E.g. Bitcoin, BitTorrent, napster etc.
\end{itemize}

\subsection{Event based architecture}
The callbacks mechanism describes this architecture well. It consists of source, listener, and a bus. The event source send a message i.e. an event on the bus to other component. The listener responds by performing some action. The components comminicate only via the event bus. The linux device driver interrupt handlers employ this mechanism. 
\begin{itemize}
\item Changing name or type of an even requires changes to the listener.
\item Too many event sources or listeners lead to bottleneck for the bus.
\item Following the control flow is difficult due to asynchronous nature of events.
\item The producer and consumer of the event need not be aware of each other allowing for loose coupling
\item Loose coupling makes it easy for a component to evolve independently.
\item Message passing over the bus introduce additional abstraction and may not be efficient.
\end{itemize}

\subsection{Interpreter}
It consists of a program to be executed, an interpreter which runs such a program, program state and memory component required any storage needs. The interpreted languages like Awk, Perl, Python or Java are prime example of this architectural style of development.
\begin{itemize}
\item Program development becomes easy with the help of interpreter.
\item Allows portability across various platforms.
\item Debugging become easier.
\item An extra level of indirection makes execution slower.
\item Interpreter can gaurd against malicious instruction 
\end{itemize}

\subsection{Blackboard}
It is similar to boardroom where people solve the problem using a whiteboard. The component blackboard acts as a global information store. It allows various components to collaborate towards the final solution. The controller component monitors the blackboard and schedules individual knowledge sources. The knowledge sources are specialized workers with its own representation of the problem. The communication happens through the blackboard. The CAD software is an example of blackboard design.
\begin{itemize}
\item Efficient scheduling of tasks and resource management across a distributed network.
\item Better suited when a problem can be split into multiple sub-problems.
\item Not always easy to break down a task into subproblems.
\item Everything is shared and can cause unwated information flows.
\item Controller design can become overly complex and unmaintainable.
\item Knowledge sources being independent, allows for reusability.
\end{itemize}

\subsection{Publish Subscribe Architecture}
\subsection{Micro Kernel Architecture}

\section{New architectural styles}
The monolithic applications causes lot of bloated code with dependencies and slower deployment cycles. This also leads to more buggy applications.
All such issues leads to newer architectural paradigm shift, which is described in the following subsections. Following patterns are discussed exclusively in this context.
\begin{itemize}
\item Service Oriented architecture (SOA)
\item Microservices
\item Containers
\item Reactive programming
\item Resilient software development
\end{itemize}

\subsection{SOA}
SOA \cite{SOA} allows application to be modularized into services. A centralized component is responsible for the service integration and comminication.
This is achieved using an Enterprise service bus (ESB), which acts as an \textit{orchestrator}. The data model for SOA applications evolve by an agreement between many services.
However, such a canonical data model being shared by many services, which limits the evolution. The services often reflect the comminication pattern of the organization, a.k.a. Conway's law \cite{conway}. \par 
The SOA is often implemented as web services which use following communication methods.
\begin{itemize}
\item Simple Object Access Protocol (SOAP) for synchronous communication with other web services, typically over HTTP. Components that use SOAP are often described in a Web Services Description Language (WSDL). The WSDL is an XML-based interface description language that is used for describing the functionality offered by a web service. 
\item The REpresentational State Transfer (REST) protocol too can be used for basic request/response operations between the services.
\end{itemize}

Most organizations need to deal with existing legacy applications before adopting the SOA architecture. The key bussiness processes depend on such legacy systems and hence a step by step solution must be adopted to move towards SOA. Following are some of the options strategies to deal with the change.
\begin{itemize}
\item Use commercial off the shelf components to replace existing system. This can be more costly in future.
\item Wrap the current legacy system with a middleware that can offer the legacy system interface through a Web service.
\item Redevelop the legacy application to achieve the optimal levels of decoupling. However, legacy code can be complex and lack of documentation could make it more challenging.
\end{itemize}
SOA sometimes is referred to as "simple service and smart pipes" \cite{comparison}.
\subsection{Microservices}
Microservices \cite{microservice} allows applications to be made up of small, self-contained independent units i.e., services working together through APIs interface. Martin fowler \cite{microservice} describes following key characteristics for of the microservice.
\begin{itemize}
\item \textit{Componentization} : Services are independent units that communicate with other services using web requests or remote procedure calls.
\item \textit{Business capabilities} :  Services represent the business capability, with responsibility for a complete stack.
\item \textit{Product rather than projects} : Microservices treat services as products and claim ownership. The ongoing maintainance requires complying with business needs.
\item \textit{Dumb pipes and smart endpoints} : Microservices do not use ESBs. The services are intelligent and communication channel is userd merely for message passing. The communicationnication channel does not implement any service functionality i.e., the channel implementation is simple.
\item \textit{Decentralized governance} : There is no orchestrator, instead it evolves towards choreagraphy and event collaboration.
\item \textit{Bounded context} : Domain-driven design complex domain up into multiple bounded contexts and maps out the relationships between them. This leads each service to have its own localized version of data model promoting loose coupling.
\end{itemize}

Microservices architecture constatly introduce new services that requires DevOps to ensure component updates in production environment. 
It has disadvantages such as code duplication, interfaces mismatch, operations overhead and the challenge of continuous testing of multiple systems. However, the benefits of creating loosely coupled components by independent teams using a variety of languages and tools far outweigh the disadvantages. Some of examples of microservice adoption are listed below.

\begin{itemize}
\item \textit{Netflix} : Moved away from single war file to full fledged microservice approach.
\item \textit{The gaurdian} : This website maintains their core monolithic system but use microservice for new components.
\item \textit{Gilt groupe} : Uses massive microservice approach to their shopping site, allowng various teams working on different services.
\end {itemize}

\subsection{Containers}
Containers \cite{lxc} \cite{kubernetes} are often termed as light weight virtual machines. An example is lxc containers built for linux systems. They provides an isolated environment with exclusive access to resources like memory, network, storage etc. The following components enable the containers to provide isolation and bare-metal performance.
\begin{itemize}
\item \textit{Kernel namespaces} --- A namespace wraps a global system resource making it appear to the processes as their own isolated instance.
\item \textit{SELinux} --- provides access control
\item \textit{SeComp} --- restrict malicious usage of system calls
\item \textit{chroot} --- allows process to change root i.e., restricts process to specific directory structure.
\item \textit{Kernel capabilities} --- Distinguish between priviledged and un-priviledged process and allow access accordingly
\item \textit{CGroups} --- Limits, accounts for and isolates process resource usage.
\end{itemize}

The virtual machines (VM) are widely used to host the applications. A single physical machine can have multiple VMs running allowing for a service per instance. The virtual machines have their own operating system (OS) and are resource heavy. Contrast this with containers which share the kernel with the host OS and provide resource isolation equivalent to VMs. This makes containers more attractive to deploy the microservices. Container can be equipped to have its own IP address allowing it to be the best option in cloud environments like Amazon EC2. Nothing restricts one to create containers insides a VM hosted on a third party cloud. \par
The containerized applications need to account for failure too. It should have a mechanism to recover from failures, i.e., restarting the service on a different host or respawing the containers. 
Today the distributed system development, with microservice architectures built from containerized software components \cite{containerPatterns} is dominating the industry.

\subsection{Reactive programming}
Reactive programming (RP) is a new paradigm that deals with the data flows and the propagation of change. It is possible to express static or dynamic data flows with ease in the reactive programming languages, and that the underlying execution model will automatically propagate changes through the data flow. For example, in a model–view–controller (MVC) architecture, RP can facilitate changes in an underlying model that are reflected automatically in an associated view \cite{mvc}. \par
Traditionally such a task was achieved using observer pattern \cite{gof}. This pattern is mainly used to implement distrubuted event driven systems, consisting following characteristics.
\begin{itemize}
\item A "subject" and "observer" are the main objects.
\item Any change to subject is automatically propagated to the observers.
\item Observers should be registered with the subject to receive the change notification.
\item Subject maintains list of observers and calls update method on all of them in the event of change.
\end {itemize}
The observer pattern is criticized for lack of composability, inversion of the logical relation among reactive entities and limited readability \cite{observer}. Also, it can be a source of memory leaks especially when the observer object fails to unregister itself. During such a situation the observer object can not be garbage collected. \par

The following pseudocode \cite{reactive_walkthrough} illustrates the reactive paradigm well. This code snippet receives the mouse.clicked event and the current mouse.position .
\begin{enumerate}
\item clicked: Event = mouse.clicked
\item scaledPosition: Signal[(Int,Int)] = mouse.position X 0.5
\item lastClick: Signal[(Int,Int)] = \\
      clicked.snapshot(scaledPosition)
\end{enumerate}

The mouse position is scaled by 0.5 and saved to scaledPosition variable. On every mouse click, the snapshot of the position is saved to lastClick. Any change to mouse position will reflect to scaled position in conventional imperative programming language. Same is true about the lastClick variable. However, RP converts the line #2 into a constraint. The RP runtime identifies the dependency between scaledPosition and mouse.position, making realtime updates possible. This technique can be applied to MVC architecture to get the view updated every time something changes in the model. 
\begin{itemize}
\item There are no bugs because of forgotten updates
\item No redundant computations in case programmers code defensively and update too much,
\item Applications are easily extensible as constraints can be composed, i.e., built on top of other constraints.
\end{itemize}

\subsection{Resilient software development}
With the advent of cloud computing, microservices and containers technology, applications are growing exponentially. The providers need to account node failure is commonplace thing for such a large scale applications. The application failure directly translate into loss of revenue. Application developer must consider a good balance between the cost of being resilient \cite{Hornsby} and the possible loss of revenue. The resilience can be achieved using some of the following patterns, which are becoming more common place in software architectures nowadays.
\par \textit{Redundancy} : Increasing the availability by deploying several instances, possibly in different zones or regions, e.g. AWS availability zones.
\par \textit{Autoscaling} : Scaling the resources dynamically according to the demand as opposed to doing it manually is auto scaling. E.g. Amazon's gold AMI are pre-configured instances which can be deployed on demand. Another example is Dockerfile which helps get rid of configuration scripts and deploy containerized application automatically.
\par \textit{Infrastructure as code} : It is process of managing the computer data centers using configuration templates. It is also referred to as Infrastructure as a Service (IaaS). In the event like accidental deletion, templates provide quick way to recover the infrastructure. 

\par \textit{Immutable infrastructure} : Immutable components are replaced for every deployment, rather than being updated in place.
\begin{itemize}
\item No updates should ever be performed on live systems
\item Always start from a new instance of the resource being provisioned
\end{itemize}
Immutable infrastructure promotes canary deployment i.e., deployement to a smallest possible set. If the such deployement fails, immediate rollback is possible.

\par \textit{Stateless application} :
Autoscaling and immutable infrastructure requires the application to be stateless. The application must treat all the client requests independent of earlier requests. It should not store any information on disk or in memory. However, state sharing within the autoscaling group is conducted using in-memory caching mechanism like Redis, Memchached etc.

\par \textit{Cascading failures} :
A single failure can trigger multiple others failures impacting the availability. E.g. if an overloaded cluster goes down, the traffic would be redirected to another cluster. This in turn overloads the new cluster, causing more failures. Following techniques are employed to handle such failures.
\begin{itemize}
\item \textit{Timing out} faster may mean the service is degraded instead of failing.
\item Due to transient errors, clients may resend the same requests. Making such requests \textit{idempotent} can solve the failures.
\item \textit{Degradation} implies that instead of failing, your application degrades to a lower-quality service. It consist of service that is easier to compute and deliver to the user.
\item \textit{Dropping} un-important traffic should help certain failure cases.
\end{itemize}

\begin{thebibliography}{00}
\bibitem{microservice} J. Lewis and M. Fowler,Microservices-A definition of this new architec-tural term, 2014.  https://martinfowler.com/articles/microservices.html
\bibitem{comparison} Cerny, Tom & J. Donahoo, Michael & Pechanec, Jiri. (2017). Disambiguation and Comparison of SOA, Microservices and Self-Contained Systems. 228-235. 10.1145/3129676.3129682.
\bibitem{SOA}Gold, N.E., Knight, C., Mohan, A., & Munro, M. (2004). Understanding service-oriented software. IEEE Software, 21, 71-77.
\bibitem{caseStudy} Marquez, Gaston & Astudillo, Hernan. (2018). Actual Use of Architectural Patterns in Microservices-based Open Source Projects.
\bibitem{conway} M. Conway, Conways law. https://en.wikipedia.org/wiki/Conway
\bibitem{hytrust} Key Control Server.  https://www.hytrust.com/products/keycontrol/
\bibitem{lxc} LxC Containers.  https://linuxcontainers.org/lxc/introduction/
\bibitem{kubernetes} Kubernetes. https://kubernetes.io/
\bibitem{containerPatterns}Brendan Burns and David Oppenheimer. 2016. Design patterns for container-based distributed systems. In Proceedings of the 8th USENIX Conference on Hot Topics in Cloud Computing (HotCloud'16). USENIX Association, Berkeley, CA, USA, 108-113.
\bibitem{mvc}Model-View-Controller and the Observer Pattern http://peak.telecommunity.com/DevCenter/Trellis#model-view-controller-and-the-observer-pattern
\bibitem{gof} Erich Gamma, Richard Helm, Ralph Johnson, John Vlissides (1994). Design Patterns: Elements of Reusable Object-Oriented Software. Addison Wesley. pp. 293ff. ISBN 0-201-63361-2.
\bibitem{observer} Maier, Ingo et al. “Deprecating the Observer Pattern.” (2010).
\bibitem{reactive_walkthrough} G. Salvaneschi, A. Margara and G. Tamburrelli, "Reactive Programming: A Walkthrough," 2015 IEEE/ACM 37th IEEE International Conference on Software Engineering, Florence, 2015, pp. 953-954. 
\bibitem{Hornsby} Adrian Hornsby, Patterns for resilient architecture, https://aws.amazon.com/developer/community/evangelists/adrian-hornsby/
\end{thebibliography}

\end{document}

%\vspace{12pt}
% \color{red}
% IEEE conference templates contain guidance text for composing and formatting conference papers. Please ensure that all template text is removed from your conference paper prior to submission to the conference. Failure to remove the template text from your paper may result in your paper not being published.


% \subsection{Units}
% \begin{itemize}
% \item Use either SI (MKS) or CGS as primary units. (SI units are encouraged.) English units may be used as secondary units (in parentheses). An exception would be the use of English units as identifiers in trade, such as ``3.5-inch disk drive''.
% \item Avoid combining SI and CGS units, such as current in amperes and magnetic field in oersteds. This often leads to confusion because equations do not balance dimensionally. If you must use mixed units, clearly state the units for each quantity that you use in an equation.
% \item Do not mix complete spellings and abbreviations of units: ``Wb/m\textsuperscript{2}'' or ``webers per square meter'', not ``webers/m\textsuperscript{2}''. Spell out units when they appear in text: ``. . . a few henries'', not ``. . . a few H''.
% \item Use a zero before decimal points: ``0.25'', not ``.25''. Use ``cm\textsuperscript{3}'', not ``cc''.)
% \end{itemize}

% \subsection{Equations}
% Number equations consecutively. To make your 
% equations more compact, you may use the solidus (~/~), the exp function, or 
% appropriate exponents. Italicize Roman symbols for quantities and variables, 
% but not Greek symbols. Use a long dash rather than a hyphen for a minus 
% sign. Punctuate equations with commas or periods when they are part of a 
% sentence, as in:
% \begin{equation}
% a+b=\gamma\label{eq}
% \end{equation}

% Be sure that the 
% symbols in your equation have been defined before or immediately following 
% the equation. Use ``\eqref{eq}'', not ``Eq.~\eqref{eq}'' or ``equation \eqref{eq}'', except at 
% the beginning of a sentence: ``Equation \eqref{eq} is . . .''

% \subsection{\LaTeX-Specific Advice}

% Please use ``soft'' (e.g., \verb|\eqref{Eq}|) cross references instead
% of ``hard'' references (e.g., \verb|(1)|). That will make it possible
% to combine sections, add equations, or change the order of figures or
% citations without having to go through the file line by line.

% Please don't use the \verb|{eqnarray}| equation environment. Use
% \verb|{align}| or \verb|{IEEEeqnarray}| instead. The \verb|{eqnarray}|
% environment leaves unsightly spaces around relation symbols.

% Please note that the \verb|{subequations}| environment in {\LaTeX}
% will increment the main equation counter even when there are no
% equation numbers displayed. If you forget that, you might write an
% article in which the equation numbers skip from (17) to (20), causing
% the copy editors to wonder if you've discovered a new method of
% counting.

% {\BibTeX} does not work by magic. It doesn't get the bibliographic
% data from thin air but from .bib files. If you use {\BibTeX} to produce a
% bibliography you must send the .bib files. 

% {\LaTeX} can't read your mind. If you assign the same label to a
% subsubsection and a table, you might find that Table I has been cross
% referenced as Table IV-B3. 

% {\LaTeX} does not have precognitive abilities. If you put a
% \verb|\label| command before the command that updates the counter it's
% supposed to be using, the label will pick up the last counter to be
% cross referenced instead. In particular, a \verb|\label| command
% should not go before the caption of a figure or a table.

% Do not use \verb|\nonumber| inside the \verb|{array}| environment. It
% will not stop equation numbers inside \verb|{array}| (there won't be
% any anyway) and it might stop a wanted equation number in the
% surrounding equation.

% \subsection{Some Common Mistakes}\label{SCM}
% \begin{itemize}
% \item The word ``data'' is plural, not singular.
% \item The subscript for the permeability of vacuum $\mu_{0}$, and other common scientific constants, is zero with subscript formatting, not a lowercase letter ``o''.
% \item In American English, commas, semicolons, periods, question and exclamation marks are located within quotation marks only when a complete thought or name is cited, such as a title or full quotation. When quotation marks are used, instead of a bold or italic typeface, to highlight a word or phrase, punctuation should appear outside of the quotation marks. A parenthetical phrase or statement at the end of a sentence is punctuated outside of the closing parenthesis (like this). (A parenthetical sentence is punctuated within the parentheses.)
% \item A graph within a graph is an ``inset'', not an ``insert''. The word alternatively is preferred to the word ``alternately'' (unless you really mean something that alternates).
% \item Do not use the word ``essentially'' to mean ``approximately'' or ``effectively''.
% \item In your paper title, if the words ``that uses'' can accurately replace the word ``using'', capitalize the ``u''; if not, keep using lower-cased.
% \item Be aware of the different meanings of the homophones ``affect'' and ``effect'', ``complement'' and ``compliment'', ``discreet'' and ``discrete'', ``principal'' and ``principle''.
% \item Do not confuse ``imply'' and ``infer''.
% \item The prefix ``non'' is not a word; it should be joined to the word it modifies, usually without a hyphen.
% \item There is no period after the ``et'' in the Latin abbreviation ``et al.''.
% \item The abbreviation ``i.e.'' means ``that is'', and the abbreviation ``e.g.'' means ``for example''.
% \end{itemize}
% An excellent style manual for science writers is \cite{b7}.

% \subsection{Authors and Affiliations}
% \textbf{The class file is designed for, but not limited to, six authors.} A 
% minimum of one author is required for all conference articles. Author names 
% should be listed starting from left to right and then moving down to the 
% next line. This is the author sequence that will be used in future citations 
% and by indexing services. Names should not be listed in columns nor group by 
% affiliation. Please keep your affiliations as succinct as possible (for 
% example, do not differentiate among departments of the same organization).

% \subsection{Identify the Headings}
% Headings, or heads, are organizational devices that guide the reader through 
% your paper. There are two types: component heads and text heads.

% Component heads identify the different components of your paper and are not 
% topically subordinate to each other. Examples include Acknowledgments and 
% References and, for these, the correct style to use is ``Heading 5''. Use 
% ``figure caption'' for your Figure captions, and ``table head'' for your 
% table title. Run-in heads, such as ``Abstract'', will require you to apply a 
% style (in this case, italic) in addition to the style provided by the drop 
% down menu to differentiate the head from the text.

% Text heads organize the topics on a relational, hierarchical basis. For 
% example, the paper title is the primary text head because all subsequent 
% material relates and elaborates on this one topic. If there are two or more 
% sub-topics, the next level head (uppercase Roman numerals) should be used 
% and, conversely, if there are not at least two sub-topics, then no subheads 
% should be introduced.

% \subsection{Figures and Tables}
% \paragraph{Positioning Figures and Tables} Place figures and tables at the top and 
% bottom of columns. Avoid placing them in the middle of columns. Large 
% figures and tables may span across both columns. Figure captions should be 
% below the figures; table heads should appear above the tables. Insert 
% figures and tables after they are cited in the text. Use the abbreviation 
% ``Fig.~\ref{fig}'', even at the beginning of a sentence.

% \begin{table}[htbp]
% \caption{Table Type Styles}
% \begin{center}
% \begin{tabular}{|c|c|c|c|}
% \hline
% \textbf{Table}&\multicolumn{3}{|c|}{\textbf{Table Column Head}} \\
% \cline{2-4} 
% \textbf{Head} & \textbf{\textit{Table column subhead}}& \textbf{\textit{Subhead}}& \textbf{\textit{Subhead}} \\
% \hline
% copy& More table copy$^{\mathrm{a}}$& &  \\
% \hline
% \multicolumn{4}{l}{$^{\mathrm{a}}$Sample of a Table footnote.}
% \end{tabular}
% \label{tab1}
% \end{center}
% \end{table}

% \begin{figure}[htbp]
% \centerline{\includegraphics{fig1.png}}
% \caption{Example of a figure caption.}
% \label{fig}
% \end{figure}

% Figure Labels: Use 8 point Times New Roman for Figure labels. Use words 
% rather than symbols or abbreviations when writing Figure axis labels to 
% avoid confusing the reader. As an example, write the quantity 
% ``Magnetization'', or ``Magnetization, M'', not just ``M''. If including 
% units in the label, present them within parentheses. Do not label axes only 
% with units. In the example, write ``Magnetization (A/m)'' or ``Magnetization 
% \{A[m(1)]\}'', not just ``A/m''. Do not label axes with a ratio of 
% quantities and units. For example, write ``Temperature (K)'', not 
% ``Temperature/K''.

% \section*{Acknowledgment}

% The preferred spelling of the word ``acknowledgment'' in America is without 
% an ``e'' after the ``g''. Avoid the stilted expression ``one of us (R. B. 
% G.) thanks $\ldots$''. Instead, try ``R. B. G. thanks$\ldots$''. Put sponsor 
% acknowledgments in the unnumbered footnote on the first page.

% \section*{References}

% Please number citations consecutively within brackets \cite{microservice}. The 
% sentence punctuation follows the bracket \cite{b2}. Refer simply to the reference 
% number, as in \cite{b3}---do not use ``Ref. \cite{b3}'' or ``reference \cite{b3}'' except at 
% the beginning of a sentence: ``Reference \cite{b3} was the first $\ldots$''

% Number footnotes separately in superscripts. Place the actual footnote at 
% the bottom of the column in which it was cited. Do not put footnotes in the 
% abstract or reference list. Use letters for table footnotes.

% Unless there are six authors or more give all authors' names; do not use 
% ``et al.''. Papers that have not been published, even if they have been 
% submitted for publication, should be cited as ``unpublished'' \cite{b4}. Papers 
% that have been accepted for publication should be cited as ``in press'' \cite{b5}. 
% Capitalize only the first word in a paper title, except for proper nouns and 
% element symbols.

% For papers published in translation journals, please give the English 
% citation first, followed by the original foreign-language citation \cite{b6}.
